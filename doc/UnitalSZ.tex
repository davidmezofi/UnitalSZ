% generated by GAPDoc2LaTeX from XML source (Frank Luebeck)
\documentclass[a4paper,11pt]{report}

\usepackage{a4wide}
\sloppy
\pagestyle{myheadings}
\usepackage{amssymb}
\usepackage[utf8]{inputenc}
\usepackage{makeidx}
\makeindex
\usepackage{color}
\definecolor{FireBrick}{rgb}{0.5812,0.0074,0.0083}
\definecolor{RoyalBlue}{rgb}{0.0236,0.0894,0.6179}
\definecolor{RoyalGreen}{rgb}{0.0236,0.6179,0.0894}
\definecolor{RoyalRed}{rgb}{0.6179,0.0236,0.0894}
\definecolor{LightBlue}{rgb}{0.8544,0.9511,1.0000}
\definecolor{Black}{rgb}{0.0,0.0,0.0}

\definecolor{linkColor}{rgb}{0.0,0.0,0.554}
\definecolor{citeColor}{rgb}{0.0,0.0,0.554}
\definecolor{fileColor}{rgb}{0.0,0.0,0.554}
\definecolor{urlColor}{rgb}{0.0,0.0,0.554}
\definecolor{promptColor}{rgb}{0.0,0.0,0.589}
\definecolor{brkpromptColor}{rgb}{0.589,0.0,0.0}
\definecolor{gapinputColor}{rgb}{0.589,0.0,0.0}
\definecolor{gapoutputColor}{rgb}{0.0,0.0,0.0}

%%  for a long time these were red and blue by default,
%%  now black, but keep variables to overwrite
\definecolor{FuncColor}{rgb}{0.0,0.0,0.0}
%% strange name because of pdflatex bug:
\definecolor{Chapter }{rgb}{0.0,0.0,0.0}
\definecolor{DarkOlive}{rgb}{0.1047,0.2412,0.0064}


\usepackage{fancyvrb}

\usepackage{mathptmx,helvet}
\usepackage[T1]{fontenc}
\usepackage{textcomp}


\usepackage[
            pdftex=true,
            bookmarks=true,        
            a4paper=true,
            pdftitle={Written with GAPDoc},
            pdfcreator={LaTeX with hyperref package / GAPDoc},
            colorlinks=true,
            backref=page,
            breaklinks=true,
            linkcolor=linkColor,
            citecolor=citeColor,
            filecolor=fileColor,
            urlcolor=urlColor,
            pdfpagemode={UseNone}, 
           ]{hyperref}

\newcommand{\maintitlesize}{\fontsize{50}{55}\selectfont}

% write page numbers to a .pnr log file for online help
\newwrite\pagenrlog
\immediate\openout\pagenrlog =\jobname.pnr
\immediate\write\pagenrlog{PAGENRS := [}
\newcommand{\logpage}[1]{\protect\write\pagenrlog{#1, \thepage,}}
%% were never documented, give conflicts with some additional packages

\newcommand{\GAP}{\textsf{GAP}}

%% nicer description environments, allows long labels
\usepackage{enumitem}
\setdescription{style=nextline}

%% depth of toc
\setcounter{tocdepth}{1}





%% command for ColorPrompt style examples
\newcommand{\gapprompt}[1]{\color{promptColor}{\bfseries #1}}
\newcommand{\gapbrkprompt}[1]{\color{brkpromptColor}{\bfseries #1}}
\newcommand{\gapinput}[1]{\color{gapinputColor}{#1}}


\begin{document}

\logpage{[ 0, 0, 0 ]}
\begin{titlepage}
\mbox{}\vfill

\begin{center}{\maintitlesize \textbf{ UnitalSZ \mbox{}}}\\
\vfill

\hypersetup{pdftitle= UnitalSZ }
\markright{\scriptsize \mbox{}\hfill  UnitalSZ  \hfill\mbox{}}
{\Huge \textbf{ Algorithms and library of abstract unitals and their embeddings \mbox{}}}\\
\vfill

{\Huge  0.29 \mbox{}}\\[1cm]
{ 19 February 2018 \mbox{}}\\[1cm]
\mbox{}\\[2cm]
{\Large \textbf{ G{\a'a}bor P{\a'e}ter Nagy\\
    \mbox{}}}\\
{\Large \textbf{ D{\a'a}vid Mez{\H o}fi\\
    \mbox{}}}\\
\hypersetup{pdfauthor= G{\a'a}bor P{\a'e}ter Nagy\\
    ;  D{\a'a}vid Mez{\H o}fi\\
    }
\end{center}\vfill

\mbox{}\\
{\mbox{}\\
\small \noindent \textbf{ G{\a'a}bor P{\a'e}ter Nagy\\
    }  Email: \href{mailto://nagyg@math.u-szeged.hu} {\texttt{nagyg@math.u-szeged.hu}}\\
  Homepage: \href{http://www.math.u-szeged.hu/~nagyg} {\texttt{http://www.math.u-szeged.hu/\texttt{\symbol{126}}nagyg}}\\
  Address: \begin{minipage}[t]{8cm}\noindent
 H-6720 Szeged, Aradi v{\a'e}rtan{\a'u}k tere 1\\
 \end{minipage}
}\\
{\mbox{}\\
\small \noindent \textbf{ D{\a'a}vid Mez{\H o}fi\\
    }  Email: \href{mailto://mezofi@math.u-szeged.hu} {\texttt{mezofi@math.u-szeged.hu}}\\
  Homepage: \href{http://www.math.u-szeged.hu/~mezofi} {\texttt{http://www.math.u-szeged.hu/\texttt{\symbol{126}}mezofi}}\\
  Address: \begin{minipage}[t]{8cm}\noindent
 H-6720 Szeged, Aradi v{\a'e}rtan{\a'u}k tere 1\\
 \end{minipage}
}\\
\end{titlepage}

\newpage\setcounter{page}{2}
\newpage

\def\contentsname{Contents\logpage{[ 0, 0, 1 ]}}

\tableofcontents
\newpage

     
\chapter{\textcolor{Chapter }{Abstract unitals}}\label{Chapter_Abstract_unitals}
\logpage{[ 1, 0, 0 ]}
\hyperdef{L}{X875A7A347D46F6FB}{}
{
  
\section{\textcolor{Chapter }{Global functions for internal usage}}\label{Chapter_Abstract_unitals_Section_Global_functions_for_internal_usage}
\logpage{[ 1, 1, 0 ]}
\hyperdef{L}{X79CB9BDF7CD35881}{}
{
  

\subsection{\textcolor{Chapter }{AU{\textunderscore}UnitalBlistList{\textunderscore}axiomcheck}}
\logpage{[ 1, 1, 1 ]}\nobreak
\hyperdef{L}{X869410DB81D004F4}{}
{\noindent\textcolor{FuncColor}{$\triangleright$\enspace\texttt{AU{\textunderscore}UnitalBlistList{\textunderscore}axiomcheck({\mdseries\slshape bmat})\index{AUUnitalBlistListaxiomcheck@\texttt{AU{\textunderscore}}\-\texttt{Unital}\-\texttt{Blist}\-\texttt{List{\textunderscore}axiomcheck}}
\label{AUUnitalBlistListaxiomcheck}
}\hfill{\scriptsize (function)}}\\
\textbf{\indent Returns:\ }
\texttt{true} if \mbox{\texttt{\mdseries\slshape bmat}} is the blist list of an abstract unital. 



 Each row of \mbox{\texttt{\mdseries\slshape bmat}} corresponds to a block of the unital. We check the sizes of the blocks and the
sizes of the intersections of the dual blocks. }

 

\subsection{\textcolor{Chapter }{AU{\textunderscore}IsUnitalBlistList}}
\logpage{[ 1, 1, 2 ]}\nobreak
\hyperdef{L}{X82985A2E796D7C37}{}
{\noindent\textcolor{FuncColor}{$\triangleright$\enspace\texttt{AU{\textunderscore}IsUnitalBlistList({\mdseries\slshape bmat})\index{AUIsUnitalBlistList@\texttt{AU{\textunderscore}}\-\texttt{Is}\-\texttt{Unital}\-\texttt{Blist}\-\texttt{List}}
\label{AUIsUnitalBlistList}
}\hfill{\scriptsize (function)}}\\
\textbf{\indent Returns:\ }
\texttt{true} if \mbox{\texttt{\mdseries\slshape bmat}} is the blist list of an abstract unital. 



 Each row of \mbox{\texttt{\mdseries\slshape bmat}} corresponds to a block of the unital. We check the sizes of the blocks and the
sizes of the intersections of the dual blocks. Wrong \mbox{\texttt{\mdseries\slshape bmat}} matrix size drops error. }

 

\subsection{\textcolor{Chapter }{AU{\textunderscore}IsUnitalIncidenceMatrix}}
\logpage{[ 1, 1, 3 ]}\nobreak
\hyperdef{L}{X81B6FFAA844D760B}{}
{\noindent\textcolor{FuncColor}{$\triangleright$\enspace\texttt{AU{\textunderscore}IsUnitalIncidenceMatrix({\mdseries\slshape incmat})\index{AUIsUnitalIncidenceMatrix@\texttt{AU{\textunderscore}}\-\texttt{Is}\-\texttt{Unital}\-\texttt{Incidence}\-\texttt{Matrix}}
\label{AUIsUnitalIncidenceMatrix}
}\hfill{\scriptsize (function)}}\\
\textbf{\indent Returns:\ }
\texttt{true} if \mbox{\texttt{\mdseries\slshape incmat}} is the incidence matrix of an abstract unital. 



 Each row of \mbox{\texttt{\mdseries\slshape incmat}} corresponds to a block of the unital. We check the sizes of the blocks and the
sizes of the intersections of the dual blocks. Wrong \mbox{\texttt{\mdseries\slshape incmat}} matrix size drops error. }

 

\subsection{\textcolor{Chapter }{AU{\textunderscore}IsUnitalBlockDesign}}
\logpage{[ 1, 1, 4 ]}\nobreak
\hyperdef{L}{X85C6163178FBA37F}{}
{\noindent\textcolor{FuncColor}{$\triangleright$\enspace\texttt{AU{\textunderscore}IsUnitalBlockDesign({\mdseries\slshape blocklist})\index{AUIsUnitalBlockDesign@\texttt{AU{\textunderscore}}\-\texttt{Is}\-\texttt{Unital}\-\texttt{Block}\-\texttt{Design}}
\label{AUIsUnitalBlockDesign}
}\hfill{\scriptsize (function)}}\\
\textbf{\indent Returns:\ }
\texttt{true} if \mbox{\texttt{\mdseries\slshape blocklist}} is the list of blocks of an abstract unital. 



 We check the sizes of the blocks and the sizes of the intersections of the
dual blocks. Wrong number of blocks or wrong number of points (union of the
blocks in \mbox{\texttt{\mdseries\slshape blocklist}}) drops error. }

 

\subsection{\textcolor{Chapter }{AU{\textunderscore}UnitalByBlistListNC}}
\logpage{[ 1, 1, 5 ]}\nobreak
\hyperdef{L}{X85D358AD87A25E8B}{}
{\noindent\textcolor{FuncColor}{$\triangleright$\enspace\texttt{AU{\textunderscore}UnitalByBlistListNC({\mdseries\slshape bmat})\index{AUUnitalByBlistListNC@\texttt{AU{\textunderscore}}\-\texttt{Unital}\-\texttt{By}\-\texttt{Blist}\-\texttt{ListNC}}
\label{AUUnitalByBlistListNC}
}\hfill{\scriptsize (function)}}\\
\textbf{\indent Returns:\ }
The unital object corresponding to the blist list \mbox{\texttt{\mdseries\slshape bmat}}. 



 The function stores \mbox{\texttt{\mdseries\slshape bmat}} and sets the order of the unital. The function \emph{do not check} the necessary conditions (the size of bmat, the sizes of the blocks and their
intersections). }

 }

 
\section{\textcolor{Chapter }{Constructing abstract unitals}}\label{Chapter_Abstract_unitals_Section_Constructing_abstract_unitals}
\logpage{[ 1, 2, 0 ]}
\hyperdef{L}{X7FA3A66B79866C24}{}
{
  

\subsection{\textcolor{Chapter }{AbstractUnitalByBlistList}}
\logpage{[ 1, 2, 1 ]}\nobreak
\hyperdef{L}{X84B17219824CF3E0}{}
{\noindent\textcolor{FuncColor}{$\triangleright$\enspace\texttt{AbstractUnitalByBlistList({\mdseries\slshape bmat})\index{AbstractUnitalByBlistList@\texttt{AbstractUnitalByBlistList}}
\label{AbstractUnitalByBlistList}
}\hfill{\scriptsize (function)}}\\
\textbf{\indent Returns:\ }
The unital object corresponding to the blist list \mbox{\texttt{\mdseries\slshape bmat}}. 



 Each row of \mbox{\texttt{\mdseries\slshape bmat}} corresponds to a block of the unital. We check the sizes of the blocks and the
sizes of the intersections of the dual blocks. Wrong \mbox{\texttt{\mdseries\slshape bmat}} matrix size drops error. The function stores \mbox{\texttt{\mdseries\slshape bmat}} and sets the \texttt{Order} of the unital. }

 

\subsection{\textcolor{Chapter }{AbstractUnitalByDesignBlocks}}
\logpage{[ 1, 2, 2 ]}\nobreak
\hyperdef{L}{X801DFE0785B65033}{}
{\noindent\textcolor{FuncColor}{$\triangleright$\enspace\texttt{AbstractUnitalByDesignBlocks({\mdseries\slshape blocklist})\index{AbstractUnitalByDesignBlocks@\texttt{AbstractUnitalByDesignBlocks}}
\label{AbstractUnitalByDesignBlocks}
}\hfill{\scriptsize (function)}}\\
\textbf{\indent Returns:\ }
The unital object corresponding to the list of blocks \mbox{\texttt{\mdseries\slshape blocklist}}. We check the sizes of the blocks and the sizes of the intersections of the
dual blocks. Wrong number of blocks or wrong number of points (union of the
blocks in \mbox{\texttt{\mdseries\slshape blocklist}}) drops error. The function stores \texttt{bmat}, which is based on \mbox{\texttt{\mdseries\slshape blocklist}}, sets the \texttt{Order} of the unital and sets the names of the points, \texttt{PointNamesOfUnital} of the unital. 



 }

 

\subsection{\textcolor{Chapter }{AbstractUnitalByIncidenceMatrix}}
\logpage{[ 1, 2, 3 ]}\nobreak
\hyperdef{L}{X7BA110B485BCA94C}{}
{\noindent\textcolor{FuncColor}{$\triangleright$\enspace\texttt{AbstractUnitalByIncidenceMatrix({\mdseries\slshape incmat})\index{AbstractUnitalByIncidenceMatrix@\texttt{AbstractUnitalByIncidenceMatrix}}
\label{AbstractUnitalByIncidenceMatrix}
}\hfill{\scriptsize (function)}}\\
\textbf{\indent Returns:\ }
The unital object corresponding to the incidence matrix \mbox{\texttt{\mdseries\slshape incmat}}. 



 Each row of \mbox{\texttt{\mdseries\slshape incmat}} corresponds to a block of the unital. We check the sizes of the blocks and the
sizes of the intersections of the dual blocks. Wrong \mbox{\texttt{\mdseries\slshape incmat}} matrix size drops error. The function stores \texttt{bmat}, which is based on \mbox{\texttt{\mdseries\slshape incmat}} and sets the \texttt{Order} of the unital. }

 }

 
\section{\textcolor{Chapter }{Methods for abstract unitals}}\label{Chapter_Abstract_unitals_Section_Methods_for_abstract_unitals}
\logpage{[ 1, 3, 0 ]}
\hyperdef{L}{X86DD62B07AF0A93F}{}
{
  

\subsection{\textcolor{Chapter }{PointsOfUnital (for IsAbstractUnitalDesign)}}
\logpage{[ 1, 3, 1 ]}\nobreak
\hyperdef{L}{X7ED4BFFE806EA9CF}{}
{\noindent\textcolor{FuncColor}{$\triangleright$\enspace\texttt{PointsOfUnital({\mdseries\slshape u})\index{PointsOfUnital@\texttt{PointsOfUnital}!for IsAbstractUnitalDesign}
\label{PointsOfUnital:for IsAbstractUnitalDesign}
}\hfill{\scriptsize (attribute)}}\\
\textbf{\indent Returns:\ }
The range \texttt{[ 1..q\texttt{\symbol{94}}3 + 1 ]}. 



 If \mbox{\texttt{\mdseries\slshape u}} is a unital of order $q$, then \mbox{\texttt{\mdseries\slshape u}} has $q^3 + 1$ points. }

 

\subsection{\textcolor{Chapter }{BlocksOfUnital (for IsAbstractUnitalDesign)}}
\logpage{[ 1, 3, 2 ]}\nobreak
\hyperdef{L}{X801EE0DE7CA291A4}{}
{\noindent\textcolor{FuncColor}{$\triangleright$\enspace\texttt{BlocksOfUnital({\mdseries\slshape u})\index{BlocksOfUnital@\texttt{BlocksOfUnital}!for IsAbstractUnitalDesign}
\label{BlocksOfUnital:for IsAbstractUnitalDesign}
}\hfill{\scriptsize (attribute)}}\\
\textbf{\indent Returns:\ }
The blocks of the unital \mbox{\texttt{\mdseries\slshape u}}. 



 If \mbox{\texttt{\mdseries\slshape u}} is a unital of order $q$, then each block is a subset of the points of the unital with $q + 1$ points. The blocks of an abstract unital form a $2-(q^3+1,q+1,1)$ design. }

 

\subsection{\textcolor{Chapter }{PointNamesOfUnital (for IsAbstractUnitalDesign)}}
\logpage{[ 1, 3, 3 ]}\nobreak
\hyperdef{L}{X78133B917ECB1DD9}{}
{\noindent\textcolor{FuncColor}{$\triangleright$\enspace\texttt{PointNamesOfUnital({\mdseries\slshape u})\index{PointNamesOfUnital@\texttt{PointNamesOfUnital}!for IsAbstractUnitalDesign}
\label{PointNamesOfUnital:for IsAbstractUnitalDesign}
}\hfill{\scriptsize (attribute)}}\\
\textbf{\indent Returns:\ }
The names of the $q^3+1$ points of \mbox{\texttt{\mdseries\slshape u}}. 



 The names of the points of \mbox{\texttt{\mdseries\slshape u}} is a list of length $q^3+1$ of arbitrary \textsf{GAP} objects. It may be set by \texttt{SetPointNamesOfUnital}. The default is the range \texttt{[ 1..q\texttt{\symbol{94}}3 + 1 ]}. }

 

\subsection{\textcolor{Chapter }{IncidenceDigraph (for IsAbstractUnitalDesign)}}
\logpage{[ 1, 3, 4 ]}\nobreak
\hyperdef{L}{X7880F5F97CC6CDCA}{}
{\noindent\textcolor{FuncColor}{$\triangleright$\enspace\texttt{IncidenceDigraph({\mdseries\slshape u})\index{IncidenceDigraph@\texttt{IncidenceDigraph}!for IsAbstractUnitalDesign}
\label{IncidenceDigraph:for IsAbstractUnitalDesign}
}\hfill{\scriptsize (attribute)}}\\
\textbf{\indent Returns:\ }
The (bipartite) digraph constructed from the boolean incidence matrix \texttt{bmat} of the unital \mbox{\texttt{\mdseries\slshape u}}. 



 }

 

\subsection{\textcolor{Chapter }{AutomorphismGroup (for IsAbstractUnitalDesign)}}
\logpage{[ 1, 3, 5 ]}\nobreak
\hyperdef{L}{X7A90B0197EFC6179}{}
{\noindent\textcolor{FuncColor}{$\triangleright$\enspace\texttt{AutomorphismGroup({\mdseries\slshape u})\index{AutomorphismGroup@\texttt{AutomorphismGroup}!for IsAbstractUnitalDesign}
\label{AutomorphismGroup:for IsAbstractUnitalDesign}
}\hfill{\scriptsize (attribute)}}\\
\textbf{\indent Returns:\ }
The automorphism group of the unital \mbox{\texttt{\mdseries\slshape u}}. 



 The function computes the automorphism group of \mbox{\texttt{\mdseries\slshape u}} with the help of its incidence digraph. }

 

\subsection{\textcolor{Chapter }{Isomorphism (for IsAbstractUnitalDesign, IsAbstractUnitalDesign)}}
\logpage{[ 1, 3, 6 ]}\nobreak
\hyperdef{L}{X841EF6D47D298FA3}{}
{\noindent\textcolor{FuncColor}{$\triangleright$\enspace\texttt{Isomorphism({\mdseries\slshape u1, u2})\index{Isomorphism@\texttt{Isomorphism}!for IsAbstractUnitalDesign, IsAbstractUnitalDesign}
\label{Isomorphism:for IsAbstractUnitalDesign, IsAbstractUnitalDesign}
}\hfill{\scriptsize (operation)}}\\
\textbf{\indent Returns:\ }
An isomorphism between the unitals \mbox{\texttt{\mdseries\slshape u1}} and \mbox{\texttt{\mdseries\slshape u1}} if they are isomorphic, and \texttt{fail} otherwise. 



 The isomorphism is a permutation which sends the points of the unital \mbox{\texttt{\mdseries\slshape u1}} to the points of the unital \mbox{\texttt{\mdseries\slshape u2}} such that the it preserves the incidence between the points and the blocks.
The function computes the isomorphism with the help of the incidence digraphs
of the unitals \mbox{\texttt{\mdseries\slshape u1}} and \mbox{\texttt{\mdseries\slshape u2}}. }

 }

 }

   
\chapter{\textcolor{Chapter }{Libraries and classes of abstract unitals}}\label{Chapter_Libraries_and_classes_of_abstract_unitals}
\logpage{[ 2, 0, 0 ]}
\hyperdef{L}{X83223EF97BFFD950}{}
{
  
\section{\textcolor{Chapter }{Classes of abstract unitals}}\label{Chapter_Libraries_and_classes_of_abstract_unitals_Section_Classes_of_abstract_unitals}
\logpage{[ 2, 1, 0 ]}
\hyperdef{L}{X862FDB387AA2BECC}{}
{
  

\subsection{\textcolor{Chapter }{HermitianAbstractUnital}}
\logpage{[ 2, 1, 1 ]}\nobreak
\hyperdef{L}{X838B7FC8821DD458}{}
{\noindent\textcolor{FuncColor}{$\triangleright$\enspace\texttt{HermitianAbstractUnital({\mdseries\slshape q})\index{HermitianAbstractUnital@\texttt{HermitianAbstractUnital}}
\label{HermitianAbstractUnital}
}\hfill{\scriptsize (function)}}\\
\textbf{\indent Returns:\ }
The classical unital object, which is the abstract unital of order \mbox{\texttt{\mdseries\slshape q}} isomorphic to the Hermitian curve in the classical projective plane. 



 The Hermitian curve has the following canonical equation: $X_0^{q + 1} + X_1^{q + 1} + X_2^{q + 1} = 0$. The function computes the blocks of the unital with the help of \texttt{PGU(3,\mbox{\texttt{\mdseries\slshape q}})} and calls \texttt{AbstractUnitalByDesignBlocks}. The \texttt{Name} of the unital is set as \texttt{HermitianAbstractUnital(\mbox{\texttt{\mdseries\slshape q}})}. }

 }

 
\section{\textcolor{Chapter }{Global functions for internal usage}}\label{Chapter_Libraries_and_classes_of_abstract_unitals_Section_Global_functions_for_internal_usage}
\logpage{[ 2, 2, 0 ]}
\hyperdef{L}{X79CB9BDF7CD35881}{}
{
  

\subsection{\textcolor{Chapter }{AU{\textunderscore}ReadLibraryDataFromFiles}}
\logpage{[ 2, 2, 1 ]}\nobreak
\hyperdef{L}{X85E7203A78843422}{}
{\noindent\textcolor{FuncColor}{$\triangleright$\enspace\texttt{AU{\textunderscore}ReadLibraryDataFromFiles({\mdseries\slshape nr, q, filename})\index{AUReadLibraryDataFromFiles@\texttt{AU{\textunderscore}}\-\texttt{Read}\-\texttt{Library}\-\texttt{Data}\-\texttt{From}\-\texttt{Files}}
\label{AUReadLibraryDataFromFiles}
}\hfill{\scriptsize (function)}}\\
\textbf{\indent Returns:\ }
The list of boolean incidence matrices of size $(q^3 + 1) \times q^2(q^2 - q + 1)$ read from \mbox{\texttt{\mdseries\slshape filename}}. 



 The file \mbox{\texttt{\mdseries\slshape filename}} must be gzipped and must contain \mbox{\texttt{\mdseries\slshape nr}} matrices of dimension mentioned above. The matrices must be 0-1 matrices
without any whitespace between the entries in one row and there must not be
any empty lines between matrices. }

 

\subsection{\textcolor{Chapter }{AU{\textunderscore}InitLibraryData}}
\logpage{[ 2, 2, 2 ]}\nobreak
\hyperdef{L}{X8014516A7CBCA8BC}{}
{\noindent\textcolor{FuncColor}{$\triangleright$\enspace\texttt{AU{\textunderscore}InitLibraryData({\mdseries\slshape })\index{AUInitLibraryData@\texttt{AU{\textunderscore}InitLibraryData}}
\label{AUInitLibraryData}
}\hfill{\scriptsize (function)}}\\
\textbf{\indent Returns:\ }




 Reads in the incidence matrices from the libraries of unitals shipped with the
package. }

 

\subsection{\textcolor{Chapter }{BBTAbstractUnital}}
\logpage{[ 2, 2, 3 ]}\nobreak
\hyperdef{L}{X7801F7627CCD8BB3}{}
{\noindent\textcolor{FuncColor}{$\triangleright$\enspace\texttt{BBTAbstractUnital({\mdseries\slshape n})\index{BBTAbstractUnital@\texttt{BBTAbstractUnital}}
\label{BBTAbstractUnital}
}\hfill{\scriptsize (function)}}\\
\textbf{\indent Returns:\ }
The \mbox{\texttt{\mdseries\slshape n}}th (abstract) unital of order 3 of the unitals by Betten, Betten and Tonchev. 



 In the paper Unitals and codes by Anton Betten, Dieter Betten and Vladimir D.
Tonchev (Discrete Mathematics 267, 2003, 23-33.) 909 unitals of order 3 were
constructed. The incidence matrices of these unitals are shipped with the
package. }

 }

 
\section{\textcolor{Chapter }{Libraries}}\label{Chapter_Libraries_and_classes_of_abstract_unitals_Section_Libraries}
\logpage{[ 2, 3, 0 ]}
\hyperdef{L}{X816D4F8581DED995}{}
{
  

\subsection{\textcolor{Chapter }{KNPAbstractUnital}}
\logpage{[ 2, 3, 1 ]}\nobreak
\hyperdef{L}{X7CCF744581EB8E44}{}
{\noindent\textcolor{FuncColor}{$\triangleright$\enspace\texttt{KNPAbstractUnital({\mdseries\slshape n})\index{KNPAbstractUnital@\texttt{KNPAbstractUnital}}
\label{KNPAbstractUnital}
}\hfill{\scriptsize (function)}}\\
\textbf{\indent Returns:\ }
The \mbox{\texttt{\mdseries\slshape n}}th (abstract) unital of order 4 of the unitals by Kr{\v c}adinac, Naki{\a'c}
and Pav{\v c}evi{\a'c}. 



 In the paper The Kramer-Mesner method with tactical decompositions: some new
unitals on 65 points by Vedran Kr{\v c}adinac, Anamari Naki{\a'c} and Mario
Osvin Pav{\v c}evi{\a'c} (Journal of Combinatorial Designs 19, 2011, 290-303.)
1777 unitals of order 4 were constructed. The incidence matrices of these
unitals are shipped with the package. }

 

\subsection{\textcolor{Chapter }{KrcadinacAbstractUnital}}
\logpage{[ 2, 3, 2 ]}\nobreak
\hyperdef{L}{X7B3CB10A79955EA7}{}
{\noindent\textcolor{FuncColor}{$\triangleright$\enspace\texttt{KrcadinacAbstractUnital({\mdseries\slshape n})\index{KrcadinacAbstractUnital@\texttt{KrcadinacAbstractUnital}}
\label{KrcadinacAbstractUnital}
}\hfill{\scriptsize (function)}}\\
\textbf{\indent Returns:\ }
The \mbox{\texttt{\mdseries\slshape n}}th (abstract) unital of order 3 of the unitals by Kr{\v c}adinac. 



 In the paper Steiner 2-designs S(2, 4, 28) with nontrivial automorphisms by
Vedran Kr{\v c}adinac (Glasnik Matemati{\v c}ki, Vol. 37 (57), 2002, 259-268.)
4466 unitals of order 3 were constructed. This library contains all the
unitals of order 3 with nontrivial automorphism group. The incidence matrices
of these unitals are shipped with the package. }

 

\subsection{\textcolor{Chapter }{AbstractUnitalLibraryInfo}}
\logpage{[ 2, 3, 3 ]}\nobreak
\hyperdef{L}{X7D89C3858378B174}{}
{\noindent\textcolor{FuncColor}{$\triangleright$\enspace\texttt{AbstractUnitalLibraryInfo({\mdseries\slshape })\index{AbstractUnitalLibraryInfo@\texttt{AbstractUnitalLibraryInfo}}
\label{AbstractUnitalLibraryInfo}
}\hfill{\scriptsize (function)}}\\
\textbf{\indent Returns:\ }




 The function prints the information about the available libraries of unitals. }

 }

 }

   
\chapter{\textcolor{Chapter }{Full points and perspectivities}}\label{Chapter_Full_points_and_perspectivities}
\logpage{[ 3, 0, 0 ]}
\hyperdef{L}{X854CDC7081F6B299}{}
{
  
\section{\textcolor{Chapter }{Full points of unitals}}\label{Chapter_Full_points_and_perspectivities_Section_Full_points_of_unitals}
\logpage{[ 3, 1, 0 ]}
\hyperdef{L}{X83BF0F62862451C9}{}
{
  

\subsection{\textcolor{Chapter }{FullPointsOfUnitalsBlocks (for IsAbstractUnitalDesign, IsPosInt, IsPosInt)}}
\logpage{[ 3, 1, 1 ]}\nobreak
\hyperdef{L}{X79CCD18E8741A288}{}
{\noindent\textcolor{FuncColor}{$\triangleright$\enspace\texttt{FullPointsOfUnitalsBlocks({\mdseries\slshape u, b1, b2})\index{FullPointsOfUnitalsBlocks@\texttt{FullPointsOfUnitalsBlocks}!for IsAbstractUnitalDesign, IsPosInt, IsPosInt}
\label{FullPointsOfUnitalsBlocks:for IsAbstractUnitalDesign, IsPosInt, IsPosInt}
}\hfill{\scriptsize (operation)}}\\
\textbf{\indent Returns:\ }
The list full point of \mbox{\texttt{\mdseries\slshape u}} w.r.t. the blocks \mbox{\texttt{\mdseries\slshape b1,b2}}. The arguments \mbox{\texttt{\mdseries\slshape b1,b2}} are either blocks of the unital \mbox{\texttt{\mdseries\slshape u}}, or indices of blocks in \texttt{BlocksOfUnital( u )}. 



 The point $P$ is a \emph{full point} of the unital $U$ w.r.t. the blocks $b_1,b_2$ if $P$ is not contained in $b_1$ or $b_2$, and, the projection with center $P$ from $b_1$ to $b_2$ is a well-defined bijection. }

 

\subsection{\textcolor{Chapter }{FullPointsOfUnitalRepresentatives (for IsAbstractUnitalDesign)}}
\logpage{[ 3, 1, 2 ]}\nobreak
\hyperdef{L}{X7BDC745F7F552133}{}
{\noindent\textcolor{FuncColor}{$\triangleright$\enspace\texttt{FullPointsOfUnitalRepresentatives({\mdseries\slshape u})\index{FullPointsOfUnitalRepresentatives@\texttt{FullPointsOfUnitalRepresentatives}!for IsAbstractUnitalDesign}
\label{FullPointsOfUnitalRepresentatives:for IsAbstractUnitalDesign}
}\hfill{\scriptsize (attribute)}}\\
\textbf{\indent Returns:\ }
A list of records \texttt{r} containing the fields \texttt{r.block1, r.block2, r.fullpts}, where \texttt{r.fullpts} is the set of full point of \mbox{\texttt{\mdseries\slshape u}} w.r.t. the blocks \texttt{r.block1, r.block2}. The returned list contains all possible full points of \mbox{\texttt{\mdseries\slshape u}} up to the automorphism group of \mbox{\texttt{\mdseries\slshape u}}. That is, if $P$ is a full point w.r.t. the blocks $b_1,b_2$, then there is an automorphism $\alpha$ of $U$ such that $P^\alpha, b_1^\alpha, b_2^\alpha$ are in the list. 



 }

 }

 
\section{\textcolor{Chapter }{Group of perspectivities}}\label{Chapter_Full_points_and_perspectivities_Section_Group_of_perspectivities}
\logpage{[ 3, 2, 0 ]}
\hyperdef{L}{X86A7B9B1794427DD}{}
{
  

\subsection{\textcolor{Chapter }{PerspectivityGroupOfUnitalsBlocks (for IsAbstractUnitalDesign, IsList, IsList, IsList)}}
\logpage{[ 3, 2, 1 ]}\nobreak
\hyperdef{L}{X807DFEFE8647525F}{}
{\noindent\textcolor{FuncColor}{$\triangleright$\enspace\texttt{PerspectivityGroupOfUnitalsBlocks({\mdseries\slshape u, b1, b2[, fullpts]})\index{PerspectivityGroupOfUnitalsBlocks@\texttt{PerspectivityGroupOfUnitalsBlocks}!for IsAbstractUnitalDesign, IsList, IsList, IsList}
\label{PerspectivityGroupOfUnitalsBlocks:for IsAbstractUnitalDesign, IsList, IsList, IsList}
}\hfill{\scriptsize (operation)}}\\
\textbf{\indent Returns:\ }
The group generated by perspectivies from block \mbox{\texttt{\mdseries\slshape b1}} to block \mbox{\texttt{\mdseries\slshape b2}} of the unital \mbox{\texttt{\mdseries\slshape u}}. Notice that the returned group consists of permutations of \texttt{[1..Order(u)+1]}. A list of full points can be given as 4th argument. It is not checked if the
elements of \mbox{\texttt{\mdseries\slshape fullpts}} are full points. 



 Perspectivities between blocks $b_1, b_2$ of an abstract unital $U$ are projections from $b_1$ to $b_2$ from a center $P$. In order the perspectivity be well-defined, $P$ must be a full point w.r.t. $b_1, b_2$. }

 }

 }

 \def\indexname{Index\logpage{[ "Ind", 0, 0 ]}
\hyperdef{L}{X83A0356F839C696F}{}
}

\cleardoublepage
\phantomsection
\addcontentsline{toc}{chapter}{Index}


\printindex

\immediate\write\pagenrlog{["Ind", 0, 0], \arabic{page},}
\newpage
\immediate\write\pagenrlog{["End"], \arabic{page}];}
\immediate\closeout\pagenrlog
\end{document}
